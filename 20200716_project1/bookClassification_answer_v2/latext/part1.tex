\section{项目描述与目标}

\noindent 文本分类作为自然语言处理领域最经典的技术之一,有着非常广泛的应用,如情感分析、情绪识别、主题分类等等。文本分类任务通常分为两大类,\textbf{单标签分类}任务和\textbf{多标签分类}任务。\texttt{单标签分类任务}指的是对于一个输入文本,我们需要输出其中的一个类别。举个例子,我们把每一篇新闻分类成一个主题(如体育或者娱乐)。相反,\texttt{多标签分类任务}指的是对于一个输入文本,输出的类别有多个,如对应一篇新闻可以同时输出多个类别:体育、娱乐和音乐。其中,单标签任务又可以分为\textbf{二元(binary)分类}和\textbf{多类别分类} ,\texttt{二元分类}指的是只有两种不同的类别。 \\

\noindent 在本项目中,我们主要来解决文本单标签的任务。数据源来自于京东电商,任务是\textit{基于图书的相关描述和图书的封面图片,自动给一个图书做类目的分类}。这种任务也叫作\texttt{多模态分类} 。在现实应用中,为了解决一个问题,我们通常会面对多种类型的数据。比如在自动驾驶,为了有效操控车辆的方向,我们可以借助于传感器数据的同时也可以借助于摄像头的数据。\\

\noindent 通过本项目的练习,你能通晓机器学习建模的各个流程:
\begin{itemize}
    \item \textbf{文本的清洗和预处理} :这是所有 NLP 项目的前提,或多或少都会用到相关的技术。
    \item \textbf{文本特征提取}:任何建模环节都需要特征提取的过程,你将会学到如何使用\texttt{tfidf}、\textt{常用的词向量}~\cite{bojanowski2017enriching}、\texttt{FastText}~\cite{mikolov2013efficient}等技术来设计文本特征。
    \item \textbf{图片特征提取}:由于项目是多模态分类,图片数据也是输入信号的一部分。你可以掌握如何通过预训练好的\texttt{CNN}~\cite{DBLP:journals/corr/Kim14f}来提取图片的特征,并且结合文本向量一起使用。
    \item \textbf{模型搭建}:在这里你将会学到如何使用各类经典的机器学习分类模型来搭建算法,其中也会涉及到各种调参等技术。除此之外,处理样本不均衡也是一个非常现实且具有挑战的问题。
    \item \textbf{结果的可视化}:很多模型目前都是黑盒子,很难去理解背后的原因。通过本项目的练习,你将有机会掌握如何对一个复杂模型的结果做一些可视化分析,试图理解背后分类正确或者分类错误的原因。 
    \item \textbf{模型的部署}:工作的最后一般都会涉及到模型的部署,在这里你将会学到如何使用 Flask 等工具来部署模型。
\end{itemize}

\noindent 同时, 通过本项目
\begin{itemize}
    \item 1.  熟练掌握分词, 过滤停止词等技术
    \item 2.  熟练掌握训练、使用\texttt{tfidf}、\textt{word2vec}、\textt{fasttext}模型
    \item 3.  熟练掌握训练\texttt{Xgboost}模型, 以及常用评价指标, 并熟练掌握\texttt{Grid Search}调参方法
    \item 4.  熟练掌握使用\texttt{Flask}部署模型
    \item 5.  了解如何处理不均衡分类问题
    \item 6.  了解如何获取词性、命名实体识别结果
    \item 7.  了解如何使用\texttt{Resnet}、\texttt{Bert}、\texttt{Xlnet}等预训练模型获取embedding
    \item 8.  了解完整的深度学习模型代码架构
    \item 9.  了解如何使用\texttt{AE}获取词嵌入表示
    \item 10. 了解什么是可解释性模型
\end{itemize}

\noindent 作为第一个项目,我们的目标是希望帮助大家能够搭建起一个比较完善的文本分类系统,之后遇到类似的任何问题,都可以有能力去攻克。